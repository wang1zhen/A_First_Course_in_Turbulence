\ifx\allfiles\undefined             %独立编译
\documentclass{article}
\usepackage{My_note}
\begin{document}
\title{}
\author{}
\date{}
%\else
%\chapter{}
\fi

\section{Introduction}

Most flows in reality are turbulent.

\subsection{The nature of turbulence}

\textbf{Irregularity}

Irregularity (randomness) makes it impossible to analyze all turbulent flows with a deterministic approach.

\textbf{Diffusivity}

Diffusivity leads to rapid mixing and high rates of transfer in momentuum, heat and mass. A flow without diffusivity is not turbulent.
Diffusivity prevents boundary-layer separation on airfoils at large angles of attack.

\textbf{Large Reynolds numbers}

In the equation of motion, the viscous terms and the nonlinear terms interact, which causes instablity, and when Reynolds number becomes large, turbulence originates.

\textbf{3D vorticity fluctuations}

Turbulence is rotational and three-dimensional, and is characterized by high levels of fluctuating vorticity. 
2D velocity fluctuations could not maintain themselves, thus two dimensional flow such as the cyclones are not turbulence themselves.

\textbf{Dissipation}

Turbulent fows are always dissipative. Deformation work is performed by viscous shear stresses, thus the internal energy is increased while the kinetic energy is decreased.
Hence, without a continuous supply of energy, the turbulence will decay very soon.
Random waves are random, but not dissipative, thus they are not turbulence.

\textbf{Continuum}

Turbulence is governed by equations of fluid mechanics, and the scales are larger than molecular length scale.

\textbf{Turbulent flow are flows}

Turbulece is a feature of fluid flows. There is not a general solution to the N-S equations, thus there is no general soluions for turbulent flow.
Every turbulent flow is different, however, they share some characteristics in common.

\subsection{Methods of analysis}

The equations of motion are too complicated, thus we cannot make quantitative predictions without empirical data.
Statistical studies could always introduce more unknowns.

Hence, we need to make some ad hoc assumptions to make the number of equations equal to the number of unknowns -- \textbf{Closure problem of turbulence theory}.

One approach: we assume that the relation between stress and rate of strain involves a turbulence-generated ``viscosity'', which plays a similar role to molecular viscosity in laminar flows.
Here we introduce ``eddy viscosity'' and ``mixing length''.

\textbf{Dimensional analysis}

Dimensional analysis is powerful in the study of turbulence when some aspects of the structure of turbulence depends only on a few independent variables or parameters.

\textbf{Asymptotic invariance}

High Reynolds number appears in turbulent flows, thus it reasonable to propose that the turbulence should behave in the limit as the Reynolds number approaches infinity.
The theory of turbulent boundary layers is an application of this method.

\textbf{Local invariance}

The characteristics of the turbulent motion at some point in time and space appear to be controlled mainly by the immediate environment (local).

As long as time scales are small enough, and the environment is changing gradually, we can assume that the turbulence is dynamically similar everywhere \textbf{if nondimensionalized with local length and time scales}.

\subsection{The origin of turbulence}

Laminar flow will turn into turbulence from instabilities (large Reynolds numbers).

Turbulence requires its environment to obtain energy so that it could maintain itself (due to dissipation). The source of energy could be shear in mean flow, or buoyancy.
If there is no source of energy, the turbulence decays, and it again turns back into laminar flow.

If we wish to maintain a laminar flow, or make a turbulent flow laminar, we shall seek for a mechanism that consumes turbulent kinetic energy.

\subsection{Diffusivity of turbulence}

Turbulent motion has an ability to transport or mix \textbf{momentum, kinetic energy}, and contaminants such as \textbf{heat, particles and moisture.}

\textbf{Diffusion in a problem with an imposed length scale}

Suppose a room with characteristic linear dimension $L$ where a heating element is installed.

$\theta$ is the temperature; $\gamma$ is the thermal diffusivity (constant).

If there is no air motion, heat has to be distributed by molecular diffusion.

\begin{equation*}
    \frac{\partial \theta}{\partial t}=\gamma\frac{\partial^2\theta}{\partial x_i \partial x_i}
\end{equation*}

Dimensionaly interpret it,

\begin{equation*}
    \frac{\Delta \theta}{T_m}\sim\gamma\frac{\Delta\theta}{L^2}
\end{equation*}

Thus,

\begin{equation*}
    T_m\sim\frac{L^2}{\gamma}
\end{equation*}

Let $L=5\ m$, $\gamma=0.20\ cm^2/sec$, we have $T_m\sim 10^6\ sec$, which is rather ineffective.

However, if there is motions, which is represented by characteristic velocity $u$, we shall have

\begin{equation*}
    T_t\sim\frac{L}{u}
\end{equation*}

To determine $T_t$, we need to find $u$. Suppose the difference in temperature is $\Delta\theta$, the buoyant acceleration could be $g\Delta\theta/\theta$, and if the height $h=0.1\ m$, we have $u^2\sim gh\Delta \theta/\theta~0.03 (m/sec)^2$, which means that $u\sim 17\ cm/sec$.

If we assume $u=5\ cm/s$ as an average throught the room and $L=5\ m$, $T_t\sim 100\ sec$.

Compared to molecular diffusion, diffusion by random motion is much more rapid,

\begin{equation*}
    \frac{T_t}{T_m}\sim\frac{L}{u}\frac{\gamma}{L^2}=\frac{\gamma}{uL}
\end{equation*}

Which is the inverse of \textbf{P\'eclet number.}

As we know, Prandtl number $Pr=\nu/\gamma$, and for gases, Prandtl number is close to $1$, we can substitute $\nu$ to $\gamma$,

\begin{equation*}
    \frac{T_t}{T_m}\sim\frac{\nu}{uL}=\frac{1}{Re}
\end{equation*}

In the example above, the Reynolds number $Re$ is about $15000$.

\textbf{Eddy diffusivity}

Due to the complication of turbulent flow, we can use a method effective diffusivity.

If we write for the diffusion of heat by turbulent motions,

\begin{equation*}
    \frac{\partial\theta}{\partial t}=K\frac{\partial^2\theta}{\partial x_i\partial x_i}
\end{equation*}

where $K$ is the \textbf{representative diffusivity (or eddy diffusivity, exchange coefficient for heat)}.

From the equation above, we can estimate a time scale,

\begin{equation*}
    T\sim\frac{L^2}{K}
\end{equation*}

Since we have the atcual time scale $T_t$ before, equating these two time scales, we have

\begin{equation*}
    K\sim uL
\end{equation*}

And finally arrive at

\begin{equation*}
    \frac{K}{\gamma}\sim\frac{K}{\nu}\sim\frac{uL}{\nu}=Re
\end{equation*}

This indicates that Reynolds number can also be interpreted as a ratio of turbulent viscosity to molecular viscosity.

N.B. eddy diffusivity $K$ is an artifice, which might not actually exist.

\textbf{Diffusion in a problem with an imposed time scale}

An example of diffusivity of turbulence, the boundary layer of the atmosphere, ommited here.

\subsection{Length scales in turbulent flows}

In turbulent flows, a wide range of length scales exists. That is why spectral analysis of turbulent motion is useful.

\textbf{Laminar boundary layers}

For steady flow of an incompressible fluid with constant viscosity, the Navier-Stokes equations are

\begin{equation*}
    u_j\frac{\partial u_i}{\partial x_j}=-\frac{1}{\rho}\frac{\partial p}{\partial x_i}+\nu\frac{\partial^2 u_i}{\partial x_j\partial x_j}
\end{equation*}

If we have $U$ as the charateristic velocity and $L$ as the charateristic length, we can estimate the inertia term as $U^2/L$, and the viscous terms are $\nu U/L^2$.
The ratio of inertia terms and viscous terms is $UL/\nu=Re$, which means that viscous terms should be negligible at large Reynolds numbers.
However, since boundary conditions and initial conditions exist, we shall not neglect viscous terms everywhere in the flow field.

As viscous terms can survive at high Reynolds numbers, we shall choose a new length scale $l$, so that the viscous terms are of the same order of magnitude as the inertia terms.

\begin{equation*}
    U^2/L\sim\nu U/l^2
\end{equation*}

thus

\begin{equation*}
    \frac{l}{L}\sim\left(\frac{\nu}{UL}\right)^{1/2}=Re^{-1/2}
\end{equation*}

Hence, we know that the viscous length $l$ is a transverse length scale, which represents the width or thickness of the boundary layer.

\textbf{Diffusive and convective length scales}

Diffusive length scale: across the flow;

Convective length scale: along the flow.

In most shear flows, their width is much smaller than their length. This is useful to some simplifying approximations. $l/L\rightarrow 0$ is a good example.


\ifx\allfiles\undefined         %独立编译
\end{document}
\fi
