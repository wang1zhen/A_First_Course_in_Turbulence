\ifx\allfiles\undefined             %独立编译
\documentclass{article}
\usepackage{My_note}
\begin{document}
\title{}
\author{}
\date{}
%\else
%\chapter{}
\fi

\section{Boundary-free shear flows}

\subsection{Almost parallel, two-dimensional flows}

Plane flows and axisymmetric flows: the mean velocity field is entirely confied to planes.

Plane flows: mean flow in planes parallel to a given plane is identical;

Axisymmetric flows: mean flow in planes through the axis of symmetry is identical.

\textbf{Plane flows}

principal mean-velocity component is in x direction, confied to the x,y plane, evolve slowly in the x direction.

\begin{equation*}
    U_i=\{U,V,0\}
\end{equation*}

\begin{equation*}
    \frac{\p}{\p x}\ll\frac{\p}{\p y}
\end{equation*}

Wake, jet and shear layer.

\begin{equation*}
    U_s=max\left|U_0-U\right|
\end{equation*}

For far wakes, $U=\mathcal{O}(U_0)$;

For jets and shear layers, $U=\mathcal{O}(U_s)$.

Generally, $U=\mathcal{O}(\tilde{U})$, $\tilde{U}=U_0$ for wakes and $\tilde{U}=U_s$ for jets and shear layers.

Cross-stream scale $\ell$,

\begin{equation*}
    \frac{\p U}{\p y}=\Order\left(\frac{U_s}{\ell}\right)
\end{equation*}

Length scale in the x direction, $L$

\begin{equation*}
    \frac{\p U}{\p x}=\Order\left(\frac{U_s}{L}\right)
\end{equation*}

Velocity scale for turbulence, $\varu$,

\begin{equation*}
    -\overline{uv}=\overline{u^2}=\overline{v^2}=\Order(\varu^2)
\end{equation*}

From the equation for continuity, $\p U/\p x+\p V/\p y=0$, the cross-stream component $V$ of the mean velocity,

\begin{equation*}
    V=\Order\left(\frac{U_s\ell}{L}\right)
\end{equation*}

\textbf{The cross-stream momentum equation}

\begin{equation*}
    U\frac{\p V}{\p x}+V\frac{\p V}{\p y}+\frac{\p}{\p x}\left(\overline{uv}\right)+\frac{\p}{\p y}\left(\overline{v^2}\right)=-\frac{1}{\rho}\frac{\p P}{\p y}+\nu\left(\frac{\p^2 V}{\p x^2}+\frac{\p^2 V}{\p y^2}\right)
\end{equation*}

We assume

\begin{equation*}
    \frac{\tilde{U}}{\varu}\frac{U_s}{\varu}\left(\frac{\ell}{L}\right)^2\rightarrow0,\quad
    \frac{U_s}{\varu}\frac{1}{R_\ell}\left(\frac{\ell}{L}\right)\rightarrow0
\end{equation*}

Identify the orders of magnitudes, the pressure term should be of the same order as $\p\overline{v^2}/\p y$. Thus,

\begin{equation*}
    \frac{\p \overline{v^2}}{\p y}=-\frac{1}{\rho}\frac{\p P}{\p y}
\end{equation*}

Integrate,

\begin{equation*}
    \frac{P}{\rho}+\overline{v^2}=\frac{P_0}{\rho}
\end{equation*}

$P_0$: the pressure outside the turbulent part of the field $(y\rightarrow\pm\infty)$. We assume $\p P_0/\p x=0$.

Take derivative w.r.t x,

\begin{equation*}
    \frac{1}{\rho}\frac{\p P}{\p x}+\frac{\p\overline{v^2}}{\p x}=0
\end{equation*}

\textbf{The streamwise momentum equation}

\begin{equation*}
    U\frac{\p U}{\p x}+V\frac{\p U}{\p y}+\frac{\p}{\p x}\left(\overline{u^2}-\overline{v^2}\right)+\frac{\p}{\p y}\left(\overline{uv}\right)=\nu\left(\frac{\p^2 U}{\p x^2}+\frac{\p^2 U}{\p y^2}\right)
\end{equation*}

Similarly, the following relation should keep:

\begin{equation*}
    \frac{\tilde{U}}{\varu}\frac{U_s}{\varu}\frac{\ell}{L}=\Order(1)
\end{equation*}

\textbf{Turbulent wakes}

If $\varu/U_s=\Order(1)$, then

\begin{equation*}
    \frac{\varu}{\tilde{U}}=\Order\left(\frac{\ell}{L}\right)
\end{equation*}

This situation occurs in far wakes. Thus, we have

\begin{equation*}
    U\frac{\p U}{\p x}+\frac{\p(\overline{uv})}{\p y}=0
\end{equation*}

For wakes, we have $\tilde{U}=U_0$, $\varu\sim U_s$,

\begin{equation*}
    \frac{U-U_0}{U_0}=\Order\left(\frac{U_s}{U_0}\right)=\Order\left(\frac{\ell}{L}\right)
\end{equation*}

Which imples we can replace $U$ by $U_0$,

\begin{equation*}
    U_0\frac{\p U}{\p x}+\frac{\p(\overline{uv})}{\p y}=0
\end{equation*}

\textbf{Turbulent jets and mixing layers}

If $\tilde{U}=U_s$, then

\begin{equation*}
    \frac{\varu}{U_s}=\Order\left(\frac{\ell}{L}\right)^{1/2}
\end{equation*}

Hence, the momentum equation

\begin{equation*}
    U\frac{\p U}{\p x}+V\frac{\p U}{\p y}+\frac{\p(\overline{uv})}{\p y}=0
\end{equation*}

\textbf{The momentum integral}

$U_0$ is not a function of position,

\begin{equation*}
    U\frac{\p (U-U_0)}{\p x}+V\frac{\p (U-U_0)}{\p y}+\frac{\p(\overline{uv})}{\p y}=0
\end{equation*}

Use the continuity equation,

\begin{equation*}
    U\frac{\p [U(U-U_0)]}{\p x}+V\frac{\p [V(U-U_0)]}{\p y}+\frac{\p(\overline{uv})}{\p y}=0
\end{equation*}

For large values of $y$, $U-U_0$ and $\overline{uv}$ vanishes,

\begin{equation*}
    \frac{\dif}{\dx}\int_{-\infty}^{\infty}U(U-U_0)\dy=0
\end{equation*}

\begin{equation*}
    \rho\int_{-\infty}^{\infty}U(U-U_0)\dy=M
\end{equation*}

\textbf{Momentum thickness}

Momentum thickness $\theta$: width of the completely separated stagnant region.

Net momentum defect per unit time and depth

\begin{equation*}
    -\rho U_0^2\theta=M
\end{equation*}

\begin{equation*}
    \theta=\int_{-\infty}^{\infty}\frac{U}{U_0}\left(1-\frac{U}{U_0}\right)\dy
\end{equation*}

It is independent of $x$.

Drag coefficient $c_d$,

\begin{equation*}
    D=c_d\frac{1}{2}\rho U_0^2d
\end{equation*}

$D$ is the drag per unit depth and $d$ is the frontal height of the obstacle.

Clearly, $D=-M$,

\begin{equation*}
    c_d=2\frac{2\theta}{d}
\end{equation*}

\subsection{Turbulent wakes}

\textbf{Self-preservation}

In wakes, we expect

\begin{equation*}
    \frac{U_0-U}{U_s}=f(\frac{y}{\ell},\frac{\ell}{L},\frac{\ell U_s}{\nu},\frac{U_s}{U_0})
\end{equation*}

Assume $\ell/L\rightarrow0, \ell U_s/\nu\rightarrow\infty, U_s/U_0\rightarrow0$, thus

\begin{equation*}
    \frac{U_0-U}{U_s}=f\left(\frac{y}{\ell}\right)
\end{equation*}

The turbulence intensity $\varu$ is of oerder $U_s$,

\begin{equation*}
    -\overline{uv}=U_s^2g\left(\frac{y}{\ell}\right)
\end{equation*}

Self-preservation hypothesis: the velocity defect and the Reynolds stress become invariant w.r.t $x$ if they are expressed in terms of the local length and velocity scales $\ell$ and $U_s$.

Define $\xi=y/\ell$,

\begin{align*}
    \frac{\p U}{\p x}&=-\frac{\dif U_s}{\dx}f+\frac{U_s}{\ell}\frac{\dif l}{\dx}\xi f'\\
    \frac{\p\overline{uv}}{\p y}&=-\frac{U_s^2}{\ell}g'
\end{align*}

(primes denote diff w.r.t $\xi$)

Thus we have,

\begin{equation*}
    -\frac{U_0\ell}{U_s^2}\frac{\dif U_s}{\dx}f+\frac{U_0}{U_s}\frac{\dif l}{\dx}\xi f'=g'
\end{equation*}

The coefficients of $f$ and $\xi f'$ must be constant, thus

\begin{equation*}
    \frac{\ell}{U_s^2}\frac{\dif U_s}{\dx}=const,\quad\frac{1}{U_s}\frac{\dif \ell}{\dx}=const
\end{equation*}

The general solution could be $\ell\sim x^n$, $U_s\sim x^{n-1}$.

The momentum integral can be written as

\begin{equation*}
    U_0U_s\ell\int_{-\infty}^{\infty}f(\xi)\dif\xi-U_s^2\ell\int_{-\infty}^{\infty}f^2(\xi)\dif\xi=-\frac{M}{\rho}
\end{equation*}

The second term should be neglected, thus we have

\begin{equation*}
    U_s\ell\int_{-\infty}^{\infty}f(\xi)\dif\xi=U_0\theta
\end{equation*}

$U_s\ell$ should be independent of $x$, thus $n=1/2$. Hereby,

\begin{equation*}
    U_s=Ax^{-1/2},\quad\ell=Bx^{1/2}
\end{equation*}

A self-preserving solution is possible only if the velocity and length scales behave as the equation above.

\textbf{The mean-velocity profile}

Substitute back into the differential equation,

\begin{equation*}
    \frac{1}{2}\left(\frac{U_0B}{A}\right)(\xi f'+f)=g'
\end{equation*}

We need a relation between $f$ and $g$.

Define eddy viscosity $\nu_T$:

\begin{equation*}
    -\overline{uv}=\nu_T\frac{\p U}{\p y}
\end{equation*}

then

\begin{equation*}
    \nu_T=-\frac{U_s\ell g}{f'}
\end{equation*}

Assume $\nu_T$ is a constant,

\begin{equation*}
    \frac{\nu_T}{U_s\ell}=\frac{1}{R_T}=-\frac{g}{f'}
\end{equation*}

$R_T$: Turbulent Reynolds number, determined by experimental data. The equation is likely to be valid only near the center line of the wake.

\begin{equation*}
    \alpha(\xi f'+f)+f''=0,\quad \alpha=\frac{1}{2}R_TU_0\frac{B}{A}
\end{equation*}

The solution is

\begin{equation*}
    f=\exp\left(-\frac{1}{2}\alpha\xi^2\right)
\end{equation*}

Since $f(0)=1$, the convention is to define $\alpha=1$, so that for $\xi=1, y=\ell$, $f\approx 0.6$.

\begin{equation*}
    \int_{-\infty}^{\infty}f(\xi)\dif\xi=(2\pi)^{1/2}
\end{equation*}

Sub back, we have

\begin{align*}
    \frac{U_s}{U_0}&=1.58\left(\frac{\theta}{x}\right)^{1/2}\\
    \frac{\ell}{\theta}&=0.252\left(\frac{x}{\theta}\right)^{1/2}
\end{align*}

This equation is sufficiently accurate.

\textbf{Axisymmetric wakes}

For axisymmetric wakes, $U_s\sim x^{-2/3}$, $\ell\sim x^{1/3}$, thus $R_\ell=U_s\ell/\nu \sim x^{-1/3}$, the structure of the axisymmetric wake is similar from the structure of the plane wake.

When $R_\ell$ is reduced to a value of the order unity, the wake ceases to be turbulent.

\begin{equation*}
    R_\ell\sim\left(\frac{U_0\theta}{\nu}\right)\left(\frac{\theta}{x}\right)^{1/3}
\end{equation*}

For moderate Reynolds numbers, $R_\ell$ reaches unity when $x/\theta$ is of order $(U_0\theta/\nu)^3$, which is a large distance.

\textbf{Scale relations}

The Reynolds stress

\begin{equation*}
    -\overline{uv}=-\frac{U_s^2 f}{R_T}
\end{equation*}

and its maximum

\begin{equation*}
    \left(-\frac{\overline{uv}}{U_s^2}\right)_{max}=(R_T^2 e)^{1/2}=0.05
\end{equation*}

Take the correlation coefficient between $u$ and $v$ as $0.4$, and we can estimate for the rms velocity fluctuation

\begin{equation*}
    \varu=\left(\frac{0.05U_s^2}{0.4}^{1/2}\right)=0.35U_s
\end{equation*}

And the wake propagate rate is

\begin{equation*}
    \frac{\dif\ell}{\dt}=U_0\frac{\dif\ell}{\dx}\approx 0.08 U_s
\end{equation*}

The ratio of two time scales $t_d/t_p\approx2$, indicates that turbulence can never be in equilibrium because it never has time to adjust to its changing environment.

\textbf{The turbulent energy budget}

The equation for the kinetic energy:

\begin{equation*}
    0=-U_0\frac{\p}{\p x}\left(\overline{\frac{1}{2}q^2}\right)-\overline{uv}\frac{\p U}{\p y}-\frac{\p}{\p y}\overline{\nu\left(\frac{1}{2}q^2+\frac{p}{\rho}\right)}-\epsilon
\end{equation*}

Here, $\overline{q^2}=\overline{u_iu_i}$, is twice the kinetic energy per unit mass.

The first term is called advection, which is the convection by the mean flow, \textbf{A}.

The second term is production, \textbf{P}.

The third term is transport by turbulent motion, \textbf{T}.

The last term is dissipation, \textbf{D}.

\textbf{TBD - ???}

\subsection{The wake of a self-propelled body}










\ifx\allfiles\undefined         %独立编译
\end{document}
\fi