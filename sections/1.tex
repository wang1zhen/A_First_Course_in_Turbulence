\section{Introduction}

Most flows in reality are turbulent.

\subsection{The nature of turbulence}

\textbf{Irregularity}

Irregularity (randomness) makes it impossible to analyze all turbulent flows with a deterministic approach.

\textbf{Diffusivity}

Diffusivity leads to rapid mixing and high rates of transfer in momentuum, heat and mass. A flow without diffusivity is not turbulent.
Diffusivity prevents boundary-layer separation on airfoils at large angles of attack.

\textbf{Large Reynolds numbers}

In the equation of motion, the viscous terms and the nonlinear terms interact, which causes instablity, and when Reynolds number becomes large, turbulence originates.

\textbf{3D vorticity fluctuations}

Turbulence is rotational and three-dimensional, and is characterized by high levels of fluctuating vorticity. 
2D velocity fluctuations could not maintain themselves, thus two dimensional flow such as the cyclones are not turbulence themselves.

\textbf{Dissipation}

Turbulent fows are always dissipative. Deformation work is performed by viscous shear stresses, thus the internal energy is increased while the kinetic energy is decreased.
Hence, without a continuous supply of energy, the turbulence will decay very soon.
Random waves are random, but not dissipative, thus they are not turbulence.

\textbf{Continuum}

