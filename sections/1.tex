\section{Introduction}

Most flows in reality are turbulent.

\subsection{The nature of turbulence}

\textbf{Irregularity}

Irregularity (randomness) makes it impossible to analyze all turbulent flows with a deterministic approach.

\textbf{Diffusivity}

Diffusivity leads to rapid mixing and high rates of transfer in momentuum, heat and mass. A flow without diffusivity is not turbulent.
Diffusivity prevents boundary-layer separation on airfoils at large angles of attack.

\textbf{Large Reynolds numbers}

In the equation of motion, the viscous terms and the nonlinear terms interact, which causes instablity, and when Reynolds number becomes large, turbulence originates.

\textbf{3D vorticity fluctuations}

Turbulence is rotational and three-dimensional, and is characterized by high levels of fluctuating vorticity. 
2D velocity fluctuations could not maintain themselves, thus two dimensional flow such as the cyclones are not turbulence themselves.

\textbf{Dissipation}

Turbulent fows are always dissipative. Deformation work is performed by viscous shear stresses, thus the internal energy is increased while the kinetic energy is decreased.
Hence, without a continuous supply of energy, the turbulence will decay very soon.
Random waves are random, but not dissipative, thus they are not turbulence.

\textbf{Continuum}

Turbulence is governed by equations of fluid mechanics, and the scales are larger than molecular length scale.

\textbf{Turbulent flow are flows}

Turbulece is a feature of fluid flows. There is not a general solution to the N-S equations, thus there is no general soluions for turbulent flow.
Every turbulent flow is different, however, they share some characteristics in common.

\subsection{Methods of analysis}

The equations of motion are too complicated, thus we cannot make quantitative predictions without empirical data.
Statistical studies could always introduce more unknowns.

Hence, we need to make some ad hoc assumptions to make the number of equations equal to the number of unknowns -- \textbf{Closure problem of turbulence theory}.

One approach: we assume that the relation between stress and rate of strain involves a turbulence-generated ``viscosity'', which plays a similar role to molecular viscosity in laminar flows.
Here we introduce ``eddy viscosity'' and ``mixing length''.

\textbf{Dimensional analysis}

Dimensional analysis is powerful in the study of turbulence when some aspects of the structure of turbulence depends only on a few independent variables or parameters.

\textbf{Asymptotic invariance}

High Reynolds number appears in turbulent flows, thus it reasonable to propose that the turbulence should behave in the limit as the Reynolds number approaches infinity.
The theory of turbulent boundary layers is an application of this method.

\textbf{Local invariance}

